\documentclass[12pt,letterpaper]{article}
\usepackage{preamble}
\usepackage{xurl}

\newcommand\course{GENETIS}
\newcommand\hwnumber{May2023}

\begin{document}
\setcounter{section}{22}
\section{Tue}
\subsection{Bash}
Learnt the following
\begin{enumerate}
  \item translate "tr": when appending with "echo foo >> bar", 
    the final character is a newline ("\n").
    Therefore the next thing we append appears on the next line.
    If we don't want this behavior, we could "tr" out the new-line character through
    \begin{verbatim}
      echo "foo" | tr "\n" "(whatever)" >> bar
    \end{verbatim}
    which writes "foo(whatever)" to the file "bar" as usual, but
    whatever we append next will go to the right of "foo(whatever)" 
    instead of to the next line.
    
  \item basic calculator "bc" for floats: \\
    "echo '' scale=2;2/100 '' | bc" returns .02 
  \item sed delimiter: turns out that any character following "s" is the delimiter,
    so for example "sed -i 's+hello+world+' file.txt" is the same as
    "sed -i 's/hello/world/' file.txt".
\end{enumerate}


\subsection{SLURM}
Went through the following documentation:
\begin{enumerate}
  \item HPC basics
  \item Storage Documentation
  \item Data Storage $\to$ Overview of File System (\textit{this one has a nice table})
  \item Batch System Concepts
\end{enumerate}


\section{Wed}
\subsection{Meeting with Julie}
\begin{enumerate}
  \item Don't get caught up in the details of the loop.
    Focus on being able to navigate through it; remember what each part does
    and what files are called to do what (and where they are). 
    \textit{Finish the first assignment by next Wednesday.}
  \item Nicholas is currently working with XF for PAEA ARA-hpol (part B), 
    so I will probably be working on the GA for ARA-hpol (part A)? 
    Confirm with Amy on Friday.
  \item Should I start looking into AraSim and learn about Birefringence? 
    Confirm with Amy on Friday.
\end{enumerate}

\subsection{SLURM}
Went over the following
\begin{enumerate}
  \item Batch System Concepts
  \item Batch Execution Environment
  \item Job Scripts\\
    \textit{seems like we don't really use parallel computing introduced in
    this section? We just submit jobs through a job array}
  \item Job Submission (mostly focused on \textbf{Job Arrays}). 
    In particular, note that for example 
    \begin{verbatim}
      sbatch --array=1-10%4 test.sh
    \end{verbatim}
    is going to submit an array of 10 jobs, but only 4 of these will be run at
    the same time (so 1-4, then 5-8, and finally 9 and 10).
\end{enumerate}
Some options:
\begin{enumerate}
  \item "A" is account ("PAS1960"); "N" is number of nodes; "t" is wall time.
  \item "n" is the number of tasks. Much better explanations available at
    \url{https://stackoverflow.com/questions/65603381/slurm-nodes-tasks-cores-and-cpus}
    and 
    \url{https://stackoverflow.com/questions/39186698/what-does-the-ntasks-or-n-tasks-does-in-slurm}
\end{enumerate}



\subsection{problems}
\begin{enumerate}
  \item still a bit unfamiliar with the SLURM directives such as "n". Should review
    the section "Job Scripts" in the future.
  \item still unfamiliar with parallel computing (multiple nodes and/or multiple cores).
    But since we always use only one node per job, I don't have to worry about
    and "srun", "sbcast" and all those parallel computing commands
    (at least not now?)
  
\end{enumerate}


\section{Thu}
\subsection{Meeting with Julie}
\begin{enumerate}
  \item received Julie's candidacy paper. Plan to start reading by next meeting (Wed).
  \item Restarted homework 1 on google drive. Half-way to finishing (?)
  \item \textbf{TODO:} Look into part E tomorrow. Seems like things are not exactly as
    described by the dissertation appendix.
\end{enumerate}



\section{Fri}
\subsection{Meeting with Julie}
\begin{enumerate}
  \item Pretty much done with homework 1 part E, but I should look into it in more
    detail once I am more familiar with the entire loop.
  \item Going through the question 3's for the remaining parts. Should be able to 
    finish on time as planned.
\end{enumerate}





\end{document}
